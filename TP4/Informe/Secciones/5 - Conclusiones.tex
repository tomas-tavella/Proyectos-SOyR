\section{Conclusiones}
    La implementación del \textit{buffer ping-pong} con distintos métodos de sincronización para la memoria compartida demostró que si bien ambos métodos son efectivos para evitar la pérdida de datos o las condiciones de carrera, las distintas opciones tienen sus ventajas y desventajas. Las colas de mensaje permiten compartir mucha mas información entre procesos para generar distintos comportamientos deseados, aunque tienen la desventaja que al eliminarse el mensaje de la cola cuando se lee, comunicar más de dos procesos se vuelve más complicado. Los semáforos tienen un funcionamiento más sencillo, que permite sincronizar más procesos con múltiples juegos de recursos compartidos, aunque su implementación no resulta tan simple en la práctica.
\\