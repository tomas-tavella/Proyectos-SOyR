\section{Resolución}

La primera tarea realizada para la resolución fue definir las direcciones que iban a ser utilizadas.\\

En el enunciado, se especifica que del proveedor se está conectado a la subred \texttt{198.235.150.128/25}, por lo tanto, se tienen 7 bits para hosts (127 posibles) y la máscara es \texttt{255.255.255.128 = 255.255.255.10000000}. La dirección asignada al router es \texttt{198.235.150.136} con \textit{Default Gateway} \texttt{198.235.150.129}.\\

Sería un error confundir la dirección asignada al router con el Default Gateway ya que esta última es la conexión hacia el proveedor y si se pone esta conexión hacia el lado del router cuando se quiera enviar un paquete de salida habría que hacer una tabla de ruteo por cada dirección de salida que se quiere tener.\\

Del lado interno se provee la subred de clase C \texttt{198.235.151.0/24}, por lo tanto, se tienen 8 bits para hosts (255 posibles) y la máscara es \texttt{255.255.255.0 = 255.255.255.00000000}. La dirección asignada al router es \texttt{198.235.151.1}.\\

Asimismo, del lado interno se tienen 5 dependencias:
\begin{itemize}
    \item \textbf{Producción} que requiere 32 hosts por lo tanto requiere 5 bits y la máscara es /27, es decir \texttt{255.255.255.224}.
    \item \textbf{Administración} que requiere 32 hosts, es igual a producción.
    \item \textbf{Gerencia} que requiere 16 hosts por lo tanto requiere 4 bits y la máscara es /28, es decir \texttt{255.255.255.240}.
    \item \textbf{Expedición} que requiere 16 hosts, es igual que gerencia.
    \item \textbf{Comedor} que requiere 64 hosts por lo tanto requiere 6 bits y la máscara es /26, es decir \texttt{255.255.255.192}.
\end{itemize}

En el edificio 1 se tiene producción y expedición lo cual significa que se necesitan 48 hosts, por lo tanto, se requieren 6 bits y la máscara es /26, es decir \texttt{255.255.255.192} (quedan 16 direcciones IP libres).\\

En el edificio 2 en cambio se encuentran gerencia, administración y comedor lo cual significa que se necesitan 112 hosts, por lo tanto, se requieren 7 bits y la máscara es /25, es decir \texttt{255.255.255.128} (quedan 16 direcciones IP libres).\\

El enunciado también dice que la red que conecta ambos edificios tiene que ser de la mínima cantidad posible por lo tanto es una máscara /30, es decir \texttt{255.255.255.252}. Esto es porque se necesita una dirección para cada router, una para el \textit{broadcast} y una para la \textit{Gateway}. \\

Luego, se calculan las subredes utilizadas. La dirección de red es \texttt{192.235.151.00000000} por lo tanto a continuación se proceden a especificar las subredes a utilizar para el edificio 1, 2 y la red de interconexión.\\

\begin{itemize}
    \item \textbf{Subred Edificio 1}: \texttt{192.235.151.10000000} 
    \begin{itemize}
        \item \textbf{Producción}: \texttt{192.235.151.110|00000}, la dirección IP es \texttt{198.235.151.192/27}.
        \item \textbf{Expedición}: \texttt{192.235.151.1000|0000}, la dirección IP es \texttt{198.235.151.128/28}.
    \end{itemize}
    \item \textbf{Subred Edificio 2}: \texttt{192.235.151.10000000} 
    \begin{itemize}
        \item \textbf{Administración}: \texttt{192.235.151. 000|00000}, la dirección IP es \texttt{198.235.151.0/27}.
        \item \textbf{Gerencia}: \texttt{192.235.151.0010|0000}, la dirección IP es \texttt{198.235.151.32/28}.
        \item \textbf{Producción}: \texttt{192.235.151. 01|000000}, la dirección IP es \texttt{198.235.151.64/26}.
    \end{itemize}
\end{itemize}

Como el edificio 1 tiene menos conexiones IP asignadas se utilizará este mismo para la red de interconexión. Esta es \texttt{192.235.151.101000|00}, la dirección IP es \texttt{198.235.151.160/30}.\\

Por último, para verificar que ninguna red esta dentro de otra, se verifica el rango de direcciones posible (es decir la dirección máxima y mínima que puede tener un host). Se resume esta información en el siguiente cuadro:\\


\begin{table}[H]
    \rowcolors{2}{white}{lightgray}
    \centering
    \begin{tabular}{c|c|c|c}
        %\hline
        \textbf{Nombre} & \textbf{Dirección de subred} & \textbf{Dirección mínima} & \textbf{Dirección máxima} \\
        \hline
        Producción & \small\texttt{192.235.151.192/27} & \small\texttt{192.235.151.193} & \small\texttt{192.235.151.223} \\ 
        %\hline
        Expedición & \small\texttt{192.235.151.128/28} & \small\texttt{192.235.151.129} & \small\texttt{192.235.151.143} \\ 
        %\hline
        Administración & \small\texttt{192.235.151.0/27} & \small\texttt{192.235.151.1} & \small\texttt{192.235.151.31} \\ 
        %\hline
        Gerencia & \small\texttt{192.235.151.32/28} & \small\texttt{192.235.151.33} & \small\texttt{192.235.151.47} \\ 
        %\hline
        Comedor & \small\texttt{192.235.151.64/26} & \small\texttt{192.235.151.65} & \small\texttt{192.235.151.127} \\ 
        %\hline
        Red de interconexión & \small\texttt{192.235.151.160/30} & \small\texttt{192.235.151.161} & \small\texttt{192.235.151.162} \\ 
        %\hline}
    \end{tabular}
    \caption{Direcciones mínimas y máximas de cada subred}
    \label{tabla_minmax}

    Una vez establecidas las subredes y conectadas en el CORE, deben setearse las tablas de ruteo para asegurar la comunicación entre los distintos hosts. El router del edificio 1 debe derivar las requests hacia las redes de gerencia, administración o comedor al router del edificio 2, y por default a la red del proveedor, y el router del edificio 2 debe derivar todas las request por default al router del edificio 1.
\end{table}