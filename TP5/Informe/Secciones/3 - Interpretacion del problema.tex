\section{Interpretación del problema}

Se debe crear el programa de un servidor concurrente que utilice el protocolo TCP, el cual aceptará conexiones de clientes que envían archivos para almacenar en el servidor. El procedimiento a seguir para la esta transferencia es el siguiente:

\begin{enumerate}\bfseries
    \item \textnormal{Una vez que un cliente establezca la conexión con el servidor, este ultimo le enviará el mensaje ``listo'' al cliente, quedando en espera del mensaje de confirmación ``archivo'' por parte del cliente.}
    \item \textnormal{Luego el servidor debe recibir un caracter \textit{espacio}, seguido del nombre para almacenar el archivo (solo caracteres alfanuméricos y un punto ``.'') y un caracter \textit{espacio} al final.}
    \item \textnormal{Una vez recibido un nombre valido, el servidor espera la recepción de un numero codificado en \textit{ascii} que indique el tamaño del archivo en bytes, seguido de un caracter \textit{espacio}.}
    \item \textnormal{Ahora, el servidor comienza la recepción de datos, los que almacena en un archivo binario con el nombre obtenido en el paso 2, hasta llegar a la cantidad de bytes obtenidos en el paso 3.}
    \item \textnormal{Como forma de confirmación, una vez terminada la recepción el servidor le enviará al cliente el siguiente mensaje:\\
     ``Archivo \enquote{\textit{Nombre de archivo}}, tamaño declarado \enquote{\textit{Cantidad de bytes declarada}} bytes, tamaño real \enquote{\textit{Cantidad de bytes recibidos}} bytes.''.}
    \item \textnormal{Finalmente, el servidor cierra la conexión con el cliente.}\\ 
\end{enumerate}

Adicionalmente, el servidor debe mantener un archivo en formato \texttt{csv} que contenga registro de:

\begin{itemize}
    \item Las conexiones entrantes.
    \item Fecha y hora de comienzo de la conexiones.
    \item Fecha y hora de finalización de la conexiones.
    \item Tamaño de los archivos recibidos.
    \item Cantidad de bytes enviados.
    \item Cantidad de bytes recibidos.\\
\end{itemize}

Para el manejo de errores se presentan varios casos de errores distintos: 

\begin{itemize}
    \item Error en el nombre de archivo y en el mensaje de confirmación ``archivo'': En estos casos, como no se llega a abrir el archivo, simplemente se envía un mensaje de error al cliente y se cierra la conexión.
    \item Si el archivo enviado es de tamaño superior al declarado, el archivo se trunca al llegar a la cantidad declarada de bytes y se cierra con un código de error, se envía un mensaje de error al cliente y se cierra la conexión.
    \item Si el archivo enviado es de tamaño inferior al declarado, debe existir un tiempo de \textit{timeout}. Una vez excedido este tiempo, si se recibió una cantidad de bytes menor a la declarada, se cierra el archivo con un código de error, se envía un mensaje de error al cliente y se cierra la conexión.
    \item Si se cierra la conexión con el cliente prematuramente, se cierra el archivo con un código de error.
\end{itemize}

Adicionalmente, en todos estos casos el código de error correspondiente se almacena en el registro que lleva el servidor.\\