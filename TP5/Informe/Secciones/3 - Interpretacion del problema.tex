\section{Interpretación del problema}

Se debe crear el programa de un servidor concurrente que utilice el protocolo TCP, para que los clientes que se conecten puedan jugar una partida de la ``Escoba de 15''. El procedimiento a seguir para el desarrollo del juego es el siguiente:

\begin{enumerate}\bfseries
    \item \textnormal{Una vez que un cliente establezca la conexión con el servidor, si es el primero, se le pregunta la cantidad de jugadores.}
    \item \textnormal{Luego, al primer cliente y a los demás que se conecten se les pide el nombre desde un proceso hijo, y estos envían un mensaje al padre indicando que el jugador está listo, quedando a la espera de la partida.}
    \item \textnormal{Cuando todos los jugadores están listos, el padre comienza a repartir las cartas, y envía los mensajes de sincronización a cada hijo para indicar que acción debe realizar.}
    \item \textnormal{Estas acciones pueden ser anunciar de quien es el turno, pedir la jugada, anunciar que jugada se realizó, o finalizar la partida y mostrar puntajes.}
    \item \textnormal{Luego de cada ronda el padre reparte nuevamente y vuelve a enviar los mensajes.}
    \item \textnormal{Finalizada la partida, se muestran los puntajes a los jugadores, y el servidor queda a la espera de una conexión nueva para comenzar un nuevo juego.}\\ 
\end{enumerate}

Adicionalmente, si el proceso padre recibe la señal \textit{SIGHUP}, espera a que termine la partida actual (o si no se está jugando, la siguiente partida), libera la memoria compartida y termina.

Para el manejo de errores se presentan varios casos de errores distintos: 

\begin{itemize}
    \item Cantidad de jugadores inválida: En estos casos simplemente se pide nuevamente que se ingrese un valor válido.
    \item Jugada inválida:  Ídem al caso anterior, se pide que ingrese un valor válido.
    \item Si la suma de las cartas es mayor a 15, o no alcanzan las cartas de la mesa, se pide que se repita la jugada.
\end{itemize}