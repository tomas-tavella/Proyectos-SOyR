\section{Interpretación del problema}

Se debe crear el programa de un servidor concurrente que utilice el protocolo TCP, el cual aceptará conexiones de clientes que envían archivos para almacenar en el servidor. El procedimiento a seguir para la esta transferencia es el siguiente:

\begin{enumerate}
    \item Una vez que un cliente establezca la conexión con el servidor, este ultimo le enviará el mensaje ``listo'' al cliente, quedando en espera del mensaje de confirmación ``archivo'' por parte del cliente.
    \item Luego el servidor debe recibir un caracter \textit{espacio}, seguido del nombre para almacenar el archivo (solo caracteres alfanuméricos y un punto ``.'') y un caracter \textit{espacio} al final.
    \item Una vez recibido un nombre valido, el servidor espera la recepción de un numero codificado en \textit{ascii} que indique el tamaño del archivo en bytes, seguido de un caracter \textit{espacio}.
    \item Ahora, el servidor comienza la recepción de datos, los que almacena en un archivo binario con el nombre obtenido en el paso 2, hasta llegar a la cantidad de bytes obtenidos en el paso 3.
    \item Como forma de confirmación, una vez terminada la recepción el servidor le enviará al cliente el siguiente mensaje:\\
     ``Archivo \enquote{\textit{Nombre de archivo}}, tamaño declarado \enquote{\textit{Cantidad de bytes declarada}} bytes, tamaño real \enquote{\textit{Cantidad de bytes recibidos}} bytes.''.
    \item Finalmente, el servidor cierra la conexión con el cliente.\\ 
\end{enumerate}

Adicionalmente, el servidor debe mantener un archivo en formato \texttt{csv} que contenga registro de:

\begin{itemize}
    \item Las conexiones entrantes.
    \item Fecha y hora de comienzo de la conexiones.
    \item Fecha y hora de finalización de la conexiones.
    \item Tamaño de los archivos recibidos.
    \item Cantidad de bytes enviados.
    \item Cantidad de bytes recibidos.\\
\end{itemize}

En tanto al manejo de errores, en todos los casos cuando se detecte un error (como por ejemplo, el nombre de archivo no cumple con las condiciones establecidas), se enviará un mensaje de error al cliente indicando el problema, y luego se cerrará la conexión.