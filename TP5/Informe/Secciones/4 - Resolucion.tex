\section{Resolución}

Se utilizaron las siguientes bibliotecas de C para poder llevar a cabo la resolución del problema planteado:
\begin{itemize}
    \item \texttt{<sys/types.h>}: biblioteca de \textit{System V} para la definicion de tipos de variables.
    \item \texttt{<sys/stat.h>}: biblioteca de \textit{System V} para la comunicación entre procesos.
    \item \texttt{<sys/socket.h>}: biblioteca de \textit{System V} para utilizar sockets.
    \item \texttt{<netinet/in.h>}: 
    \item \texttt{<arpa/inet.h>}: 
    \item \texttt{<unistd.h>}: 
    \item \texttt{<sys/time.h>}: biblioteca para obtener el tiempo de la \textit{timestamp}.
\end{itemize}

Para la simplificación del codigo se crearon 2 macros, SEND() y SEND_RECV() las cuales son utilizadas para enviar un mensaje al cliente y para enviar y recibir un mensaje del cliente respectivamente.\\
Al ser un servidor concurrente, un proceso hijo dentro del servidor toma a cada cliente y dentro de este se hace la copia del archivo.\\
Los buffers de transmisión y recepción son de 1500 bytes, es por eso que se envia esta cantidad de bytes desde el archivo que se quiere copiar hasta llegar al end of file del mismo.\\

\subsubsection{Pseudocódigo}

\begin{lstlisting}[language=C]          % No usar tildes en el pseudocódigo y dejar una linea vacía al principio y al final

    Servidor:
    INICIO
        Declarar y asignar variables, macros y estructuras;
        Se crea el descriptor del socket;
        Mientras(No se termine de leer el archivo){
            Acepta la conexion de un cliente (en el caso de no poder conectarlo imprime error);
            Toma el tiempo inicial;
            Imprime los datos del cliente conectado;
            Si(Estoy en el hijo){
                Declarar y asignar variables;
                Imprimen los datos del hijo;
                Cierra el socket que no es usado;
                Envia el mensaje "Listo" (en el caso de no poder enviarlo imprime error);
                Aumenta la variable de bytes enviados;                
                Recibe el mensaje "Archivo" (en el caso de no poder recibirlo imprime error);
                Aumenta la variable de bytes recibidos;
                Si(Se recibe el mensaje "Archivo"){
                    Imprime que se recibio la palabra archivo;
                    Recibe el nombre del archivo y los bytes que contiene (en el caso de no poder recibirlo imprime error);
                    Aumenta la variable de bytes recibidos;
                    Imprime el nombre del archivo;
                    Crea el archivo para escritura en binario;
                }
                Sino{
                    Imprime error;
                    Envia el mensaje de error (en el caso de no poder enviarlo imprime error);
                    Aumenta la variable de bytes enviados;
                    Toma el tiempo final;
                    Escribe en el archivo de registro;
                    Cierra la conexion con el cliente;
                    Termina el programa;
                }
                Hacer{
                    Recibe un buffer con la informacion contenida en el archivo y lo escribe en el nuevo archivo (en el caso de no poder recibirlo imprime error);
                    Decrementa la variable que contiene los bytes del archivo;
                    Aumenta la variable de bytes recibidos;
                    Imprime la cantidad de bytes recibidos del cliente;
                    Si(La variable que contiene los bytes del archivo es menor a 0){
                        Imprime que llegaron bytes de mas;
                        Envia el mensaje de error (en el caso de no poder enviarlo imprime error);
                        Aumenta la variable de bytes enviados;
                        Toma el tiempo final;
                        Escribe en el archivo de registro;
                        Cierra la conexion con el cliente;
                        Termina el programa;
                    }
                }Mientras(No se llegue al final del archivo)
                Cierra el archivo;
                Imprime que la recepcion finalizo sin errores
                Envia el mensaje al cliente;
                Aumenta la variable de bytes enviados;
                Cierra la conexion con el cliente;
                Toma el tiempo final;
                Escribe en el archivo de registro;
                Termina el programa;
            }
            Cierra la conexion con el cliente;
        }
        Espera que el hijo termine se ejecucion;
        Cierra la conexion con el cliente;
        Termina el programa;
    FIN

    Cliente:
    INICIO
    Declarar y asignar variables, macros y estructuras;
    Si el segundo argumento no del formato www.xxx.yyy.zzz imprime error;
    Se crea el descriptor del socket;
    Se conecta con el servidor (en caso de no poder conectarse imprime error);
    Imprime la informacion del servidor;
    Recibe el mensaje "Listo" (en el caso de no poder recibirlo imprime error);
    Aumenta la variable de bytes recibidos;
    Si(Se recibe el mensaje "Listo"){
        Imprime que se recibio la palabra listo;
        Envia el mensaje "Archivo" (en el caso de no poder enviarlo imprime error);
    }
    Sino{
        Imprime error;
        Envia el mensaje de error (en el caso de no poder enviarlo imprime error);
        Termina el programa;
    }
    Espera a que se ingrese el nombre del archivo;
    Imprime el nombre del archivo y lo abre;
    Si(No puede abrirse el archivo){
        Imprime error;
        Envia el mensaje de error (en el caso de no poder enviarlo imprime error);
        Termina el programa;
    }
    Sino{
        Obtiene el tamano del archivo;
        Envia el nombre del archivo y el tamano (en el caso de no poder enviarlo imprime error);
    }
    Espera a que se presione enter para enviar los datos;
    Mientras(No se llegue al end of file){
        Se lee parte del archivo y se envian esos datos;
    }
    Se cierra el archivo;
    Se recibe que se termino la recepcion;
    Se cierra al cliente;
    Termina el programa
}

\end{lstlisting}

\subsection{Compilación y ejecución de los programas}

Con el código completo, para compilar los programas a un archivo binario ejecutable se llama al comando \texttt{gcc} (\textit{GNU C Compiler}):

\begin{center}
    \texttt{\$ gcc servidorTCP.c -o servidorTCP}\\
    \texttt{\$ gcc clienteTCP.c -o clienteTCP}\\
\end{center}

Con lo que se obtienen dos archivos binarios ejecutables, los cuales, estando situados en la carpeta en la que se encuentran, se ejecutan desde la terminal de la siguiente manera:

\begin{center}
    \texttt{\$ ./servidor}\\
    \texttt{\$ ./cliente 127.0.0.1}\\
\end{center}

Es importante ejecutar el servidor antes que el cliente.