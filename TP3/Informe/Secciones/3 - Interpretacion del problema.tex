\section{Interpretación del problema}
Para este programa, se debe crear un espacio de memoria compartida utilizando las funciones de la biblioteca \texttt{<sys/shm.h>}, en el que existirá un \textit{buffer ping-pong}: un \textit{buffer} que se divide en dos mitades, y mientras el productor escribe en una de las mitades, el consumidor lee de la otra mitad que ya fue escrita. Al terminar la escritura y lectura, se "intercambian" las mitades y se repite el proceso.

El programa debe leer datos del archivo \texttt{datos.dat} y almacenarlos en forma de estructuras con los siguientes datos:\\

\begin{itemize}
    \item Una variable de tipo \texttt{int} que almacena un identificador menor a 50000.
    \item 
\end{itemize}

Si la señal es \texttt{SIGUSR1}, el handler debe hacer un fork y el hijo resultante debe imprimir el contenido del vector separado con comas, su PID, y el PID de su padre.\\

Si la señal es \texttt{SIGUSR2}, se debe crear un \textit{thread} o hilo que multiplique a los elementos por si mismos e imprima los valores separados por espacio, y al finalizar devuelva la suma de los elementos. No puede haber más de 10 hilos en total.\\

Además, el proceso padre tiene que imprimir el estado en el cual terminaron los hijos, verificar que no quede ninguno en estado \textit{zombie}, y avisar si se llega al máximo de hilos. Si se recibe la señal \texttt{SIGTERM}, se espera a que termine todos los hijos e hilos y se termina el programa.
