\section{Interpretación del problema}
Para este programa, se debe crear dos espacios de memoria compartida utilizando las funciones de la biblioteca \texttt{<sys/shm.h>}, en los que existirán las dos mitades de un \textit{buffer ping-pong}.\\

Los datos leídos del archivo \texttt{datos.dat} se almacenan en una estructura con los siguientes elementos:

\begin{itemize}
    \item Una variable de tipo \texttt{int} que almacena un identificador menor a 50000.
    \item Una etiqueta con el tiempo en el que fue escrito el dato, con precisión de micro segundos.
    \item El dato que se leyó del archivo, de tipo \texttt{float}.\\
\end{itemize}

Como no se aclara el tamaño del \textit{buffer} en el enunciado, se elige que este tenga un tamaño total de 100 estructuras de las mencionadas previamente, dividido en dos mitades de 50 estructuras.\\

El proceso productor debe leer datos del archivo \texttt{datos.dat} y almacenar las estructuras en su mitad del \textit{buffer}, mientras que el proceso consumidor debe leerlas en la otra mitad del \textit{buffer} a medida que estén disponibles, para imprimirlas en pantalla y escribirlas en un nuevo archivo \texttt{datos.csv}.\\

Para administrar el \textit{buffer} de manera que no surjan condiciones de carrera y tanto el productor como el consumidor puedan trabajar en secciones críticas sin interrupciones, se van a crear dos programas que utilizan métodos distintos para este fin:\\

\begin{itemize}
    \item \textbf{Semáforos:} se utilizan los semáforos para sincronizar con las llamadas a sistema contenidas en la biblioteca \texttt{<sys/sem.h>} y memoria compartida mediante \texttt{<sys/ipc.h>}.
    \item \textbf{Colas de mensajes:} se utilizan las colas de mensajes para sincronizar mediante las llamadas a sistema de la biblioteca \texttt{<sys/msg.h>}.\\
\end{itemize}