\section{Interpretación del problema}
El programa debe recibir un entero como argumento por línea de comandos, debe crear un vector de esa cantidad de elementos de tipo \textit{double} con la función \texttt{malloc()} para reservar la memoria, debe preparar los \textit{signal handlers} necesarios y quedar en espera de las señales.\\

Si la señal es \texttt{SIGUSR1}, el handler debe hacer un fork y el hijo resultante debe imprimir el contenido del vector separado con comas, su PID, y el PID de su padre.\\

Si la señal es \texttt{SIGUSR2}, se debe crear un \textit{thread} o hilo que multiplique a los elementos por si mismos e imprima los valores separados por espacio, y al finalizar devuelva la suma de los elementos. No puede haber más de 10 hilos en total.\\

Además, el proceso padre tiene que imprimir el estado en el cual terminaron los hijos, verificar que no quede ninguno en estado \textit{zombie}, y avisar si se llega al máximo de hilos. Si se recibe la señal \texttt{SIGTERM}, se espera a que termine todos los hijos e hilos y se termina el programa.
