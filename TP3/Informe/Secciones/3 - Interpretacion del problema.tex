\section{Interpretación del problema}
Para este programa, se debe crear un espacio de memoria compartida utilizando las funciones de la biblioteca \texttt{<sys/shm.h>}, en el que existirá un \textit{buffer ping-pong}: un \textit{buffer} que se divide en dos mitades, y mientras el proceso productor escribe en una de las mitades, el proceso consumidor lee de la otra mitad que ya fue escrita. Al terminar la escritura y lectura, se "intercambian" las mitades y se repite el procedimiento.\\

El proceso productor debe leer datos del archivo \texttt{datos.dat} y almacenarlos en forma de estructuras con los siguientes datos:\\

\begin{itemize}
    \item Una variable de tipo \texttt{int} que almacena un identificador menor a 50000.
    \item Una etiqueta con el tiempo en el que fue escrito el dato, con precisión de micro segundos.
    \item El dato que se leyó del archivo.\\
\end{itemize}

Mientras tanto, el proceso consumidor debe leer estas estructuras en la otra mitad del \textit{buffer} a medida que estén disponibles, para imprimirlas en pantalla y pasarlas a formato \texttt{.csv} (mediante las conversiones apropiadas) para luego escribirlas en un nuevo archivo \texttt{datos.csv}.\\

Para administrar el \textit{buffer} de manera que no surjan condiciones de carrera y tanto el productor como el consumidor puedan trabajar en secciones críticas sin interrupciones, se van a crear dos programas que utilizan métodos distintos para este fin:\\

\begin{itemize}
    \item \textbf{Semáforos y variables compartidas:} se utilizan los semáforos con las llamadas a sistema contenidas en la biblioteca \texttt{<sys/sem.h>} y variables compartidas.
    \item \textbf{Colas de mensajes:} se utilizan las colas de mensajes mediante las llamadas a sistema de la biblioteca \texttt{<sys/msg.h>}.\\
\end{itemize}
