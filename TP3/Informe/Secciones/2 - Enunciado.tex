\section{Enunciado}
Se desea escribir un programa que genere un vector de N elementos con los valores de 1 a N almacenados como double. El valor de N se pasará en la línea de comando en el momento de iniciar el programa.\\

Si el programa recibe la señal \texttt{SIGUSR1} deberá crear un proceso hijo, que imprimirá los valores almacenados en el vector separados por comas y los imprimirá en pantalla, junto con la identificación del proceso y del proceso padre.\\

Si el programa recibe la señal \texttt{SIGUSR2} deberá crear un hilo, este hilo deberá modificar cada elemento
del vector, multiplicándolo por si mismo, imprimir en pantalla los valores obtenidos separados por espacios y devolver el valor de la suma de los elementos del vector cuando termine.\\

El programa deberá llevar la cuenta de la cantidad de hijos e hilos generados, imprimir los resultados
devueltos por cada hilo, verificar que N no sea mayor que 20 y mostrar el estado con el que termina
cada hijo (PID y Estado). Una vez que se generen 10 hilos, no deberá generar más hilos cuando reciba
la señal \texttt{SIGUSR2} e imprimirá un cartel avisando que la cantidad máxima de hilos ha sido alcanzada.\\

Si el programa recibe la señal \texttt{SIGTERM}, esperará que se completen los hilos e hijos pendientes si los
hubiera y terminará.\\

Tener en cuenta las siguientes indicaciones:
\begin{itemize}
    \item Cuidar que no queden procesos en estado \textit{zombie}.
    \item No utilizar variables globales dentro de los hilos.
    \item El espacio de memoria para el vector con los N elementos debe reservarse dinámicamente
mediante la función \texttt{malloc()}.
\end{itemize}
