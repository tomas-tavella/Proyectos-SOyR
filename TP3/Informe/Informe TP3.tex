\documentclass{article}

% Language setting
% Replace `english' with e.g. `spanish' to change the document language
\usepackage[spanish]{babel}

% Set page size and margins
% Replace `letterpaper' with`a4paper' for UK/EU standard size
\usepackage[a4paper,top=2cm,bottom=2cm,left=3cm,right=3cm,marginparwidth=1.75cm]{geometry}
\usepackage{csquotes}
\usepackage[backend=biber]{biblatex}            % Package para bibliografia
\addbibresource{Bibliografia.bib}               % Llamo al archivo que contiene las referencias

% Useful packages
\usepackage{amsmath}
\usepackage{amsfonts}
\usepackage{graphicx}
\usepackage[colorlinks=true, allcolors=black]{hyperref}
\usepackage{float}                              % Permite forzar una posicion para una figura
\usepackage{svg}                                % Para poder insertar SVGs
\usepackage[justification=centering]{caption}   % Para poder centrar captions de figuras
\usepackage{gensymb}                            % Símbolo de grados (\degree)
\usepackage{subfigure}                          % Para poder usar subfiguras (logos de UNLP y FI juntos)
\usepackage{listings}                           % Para poder incluir código
\usepackage{xcolor}                             % Para poder definir colores        

\definecolor{codegreen}{rgb}{0,0.6,0}           % Colores para el codigo
\definecolor{codegray}{rgb}{0.5,0.5,0.5}
\definecolor{codepurple}{rgb}{0.58,0,0.82}
\definecolor{backcolour}{rgb}{0.95,0.95,0.92}

\lstdefinestyle{mystyle}{                       % Parametros para setear como se 
    backgroundcolor=\color{backcolour},         % muestra el codigo
    commentstyle=\color{codegreen},
    keywordstyle=\color{magenta},
    numberstyle=\tiny\color{codegray},
    stringstyle=\color{codepurple},
    basicstyle=\ttfamily\scriptsize,
    breakatwhitespace=false,         
    breaklines=true,                 
    captionpos=b,                    
    keepspaces=true,                 
    numbers=left,                    
    numbersep=5pt,                  
    showspaces=false,                
    showstringspaces=false,
    showtabs=false,                  
    tabsize=4
}

\lstset{style=mystyle}

\begin{document}

    \nocite{*}                                  % Se usa para que aparezcan todas la referencias del .bib sin tener que 
                                                % citarlas en el texto
        \begin{titlepage}
        \begin{center}
            \vspace*{0.5cm}
            \huge
            \textbf{Sistemas Operativos y Redes (E0224) \\ Año 2021}    % Titulo
            \\
            \vspace{0.5cm}
            Trabajo Práctico N°5                                        % Subtitulo
            \\
            \vspace{2cm}
            \Large
            \textbf{Grupo N°4:}
            \\
            \large
            \vspace{0.2cm}
            Ignacio Hamann - 68410/3
            \\
            Juan Pablo Elisei - 68380/5
            \\
            Tomás Tavella - 68371/4
            \\
            \vspace{2cm}
            \begin{abstract}
                En este informe se detalla la implementación de un servidor concurrente mediante el protocolo TCP, el cual acepta conexiones de clientes para almacenar archivos.
            \end{abstract}
            \vfill
            \begin{figure}[H]
                \centering
                \begin{subfigure}
                    \centering
                    \includegraphics[width=0.25\textwidth]{Imagenes/UNLP.pdf}
                \end{subfigure}
                \begin{subfigure}
                    \centering
                    \includegraphics[width=0.32\textwidth]{Imagenes/FI.jpg}
                \end{subfigure}
            \end{figure}
            \vspace{1cm}
            \textit{
            Facultad de Ingeniería
            \\
            Universidad Nacional de La Plata}
            \vspace{1cm}
        \end{center}
    \end{titlepage}

    \tableofcontents \newpage

    \section{Enunciado}
En la planta industrial de una empresa, hay cuatro dependencias: Gerencia, Producción, Administración y Expedición.\\

Se contrata un servicio para proveer internet, y se quiere diseñar la interconexión de las dependencias asignando a cada una de ellas subredes separadas, asignando 32 direcciones IP a cada una de las subredes pertenecientes a Producción y Administración, 16 direcciones IP a cada subred de Gerencia y Expedición y además se quiere proveer WiFi en el Comedor, mediante otra subred con 64 direcciones IP.\\

La planta industrial tiene dos edificios separados, en el primero de ellos se ubican Producción y Expedición y en el segundo se encuentran Administración, Gerencia y el Comedor.\\

El proveedor de servicio de Internet, instala la conexión en el primer edificio y provee un router que desde el lado externo está conectado a la subred \texttt{198.235.150.128/25} con dirección IP \texttt{198.235.150.136} con \textit{Default Gateway} tiene \texttt{198.235.150.129}. Del lado interno de la empresa, provee la subred clase C \texttt{198.235.151.0/24}, y la dirección asignada al router es \texttt{198.235.151.1}. La máscara de esta subred puede modificarse, pero no el IP del router.\\

\begin{enumerate}
    \item Proponga un esquema de conexión de las distintas subredes, que emplee un router en cada uno de los edificios, y que utilice una subred diferente para interconectar ambos routers. Esta última subred debe emplear la mínima cantidad posible de direcciones IP.
    \item Asigne números de subred y máscaras a cada subred. Enumere las direcciones de red y de broadcast de cada una de ellas, trate de que queden la mayor cantidad de direcciones libres para eventuales ampliaciones. 
    \item Asigne un \textit{Default Gateway} a cada subred.
    \item Para evitar instalar protocolos de ruteo internos, la empresa decide instalar rutas estáticas en los routers. Escriba cuales serían las tablas de ruteo necesarias en cada router, para que todos los hosts puedan alcanzar Internet, y además se puedan comunicar entre sí.  Si en su diseño de red, los routers poseen más de una interfaz, enumérelas como \texttt{IF0}, \texttt{IF1}, ... , \texttt{IFN} si necesita explicitar la interfaz de salida.
    \item Simule su diseño en \textit{CORE}:
    \begin{enumerate}
        \item En la simulación, debe mostrarse por lo menos dos hosts conectados en cada subred, excepto en el enlace entre routers.
        \item En la simulación debe ser posible  observar el funcionamiento del protocolo ARP para obtener las direcciones físicas.
        \item También debe ser posible mostrar la conectividad entre los diferentes hosts de  la red y con la salida a Internet mediante el uso del comando \texttt{ping}.
    \end{enumerate}
\end{enumerate}


    \section{Interpretación del problema}

Se debe crear el programa de un servidor concurrente que utilice el protocolo TCP, el cual aceptará conexiones de clientes que envían archivos para almacenar en el servidor. El procedimiento a seguir para la esta transferencia es el siguiente:

\begin{enumerate}
    \item Una vez que un cliente establezca la conexión con el servidor, este ultimo le enviará el mensaje ``listo'' al cliente, quedando en espera del mensaje de confirmación ``archivo'' por parte del cliente.
    \item Luego el servidor debe recibir un caracter \textit{espacio}, seguido del nombre para almacenar el archivo (solo caracteres alfanuméricos y un punto ``.'') y un caracter \textit{espacio} al final.
    \item Una vez recibido un nombre valido, el servidor espera la recepción de un numero codificado en \textit{ascii} que indique el tamaño del archivo en bytes, seguido de un caracter \textit{espacio}.
    \item Ahora, el servidor comienza la recepción de datos, los que almacena en un archivo binario con el nombre obtenido en el paso 2, hasta llegar a la cantidad de bytes obtenidos en el paso 3.
    \item Como forma de confirmación, una vez terminada la recepción el servidor le enviará al cliente el siguiente mensaje:\\
     ``Archivo \enquote{\textit{Nombre de archivo}}, tamaño declarado \enquote{\textit{Cantidad de bytes declarada}} bytes, tamaño real \enquote{\textit{Cantidad de bytes recibidos}} bytes.''.
    \item Finalmente, el servidor cierra la conexión con el cliente.\\ 
\end{enumerate}

Adicionalmente, el servidor debe mantener un archivo en formato \texttt{csv} que contenga registro de:

\begin{itemize}
    \item Las conexiones entrantes.
    \item Fecha y hora de comienzo de la conexiones.
    \item Fecha y hora de finalización de la conexiones.
    \item Tamaño de los archivos recibidos.
    \item Cantidad de bytes enviados.
    \item Cantidad de bytes recibidos.\\
\end{itemize}

En tanto al manejo de errores, en todos los casos cuando se detecte un error (como por ejemplo, el nombre de archivo no cumple con las condiciones establecidas), se enviará un mensaje de error al cliente indicando el problema, y luego se cerrará la conexión.

    \section{Resolución}

Placeholder text.

    \section{Conclusiones}a
Luego de realizar el trabajo se pudo rescatar que los semaforos tiene una relacion con la rama de comunicaciones ya que no son una opción.
\\

    \printbibliography 
    
\end{document}