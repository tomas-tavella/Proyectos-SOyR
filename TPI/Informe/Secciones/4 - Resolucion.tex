\section{Resolución}

Se utilizaron las siguientes bibliotecas de C para poder llevar a cabo la resolución del problema planteado:\\

\begin{itemize}
    \item \texttt{<sys/ipc.h>}: biblioteca de \textit{System V} para la comunicación entre procesos.
    \item \texttt{<sys/shm.h>}: biblioteca de \textit{System V} para la memoria compartida.
    \item \texttt{<sys/msg.h>}: implementación de \textit{System V} para colas de mensajes.
    \item \texttt{<sys/socket.h>}: biblioteca de \textit{System V} para utilizar sockets.
    \item \texttt{<sys/wait.h>}: biblioteca de \textit{System V} para ultilar la funcion \texttt{wait()}.
    \item \texttt{<netinet/in.h>}: biblioteca que contiene varias definiciones útiles de macros y tipos de datos para manejo de sockets de red.
    \item \texttt{<arpa/inet.h>}: biblioteca que contiene definiciones de varias funciones útiles para el manejo de direcciones IP.\\
    
\end{itemize}

Se decidió guardar los datos de cada jugador (nombre, cantidad y valor de las cartas levantadas, y escobas) y la jugada realizada (cartas levantadas o descartadas) como memoria compartida, para simplificar la comunicación entre procesos, y que no sea necesario mandar tantos datos en los mensajes entre procesos cuando se debe anunciar de quien es el turno o que jugada realizó.\\

El padre será el encargado de sincronizar la partida, repartir las cartas, y avisarle a los hijos que operación tienen que realizar. Los hijos por su parte esperarán un mensaje del padre una vez estén conectados los jugadores, y anunciarán de quien es el turno, pedirán la jugada o mostraran que jugada se realizó de acuerdo a lo recibido.\\

Una vez terminada la partida (chequeado porque se repartieron las 40 cartas y no se pueden repartir más), el padre enviará un mensaje de finalización, para que los hijos muestren los puntajes, se desasocien de la memoria compartida y terminen, y quedará a la espera de una nueva conexión para iniciar otra partida.\\

\subsection{Pseudocódigo}

A continuación se muestra un pseudocódigo que explica a grandes rasgos el funcionamiento del programa del servidor:\\

\begin{lstlisting}[language=C]          

    INICIO
        Declarar y asignar variables, macros, estructuras y funciones;
        Obtener la clave de las memorias compartidas y colas de mensajes (en el caso de no obtenerlas imprimir error);
        Llamar al sistema para obtener el ID de las memorias compartidas y colas de mensajes (en el caso de que no obtenerlas imprimir error);
        Asociar el espacio de memoria compartida con un puntero (si no puede asociar imprimir error);
        Obtener colas de mensajes (si no puede obtener imprimir error);
        Crear socket para el servidor;
        Crear partida y conexion de jugadores;
        Mientras(El servidor este activo){
            Mientras(No se hayan terminado de conectar los jugadores){
                Esperar a que se conecte un jugador;
                Imprimir desde donde se hizo la conexion;
                Si(Es el primer jugador){
                    Solicitar la cantidad de jugadores que van a jugar (si no se ingresa un numero entre 2 y 4 volver a solicitar);
                }
                Si(Se crea un hijo y se esta en el hijo){
                    Solicitar ingresar un nombre (si no ingresa se le asigna un nombre por defecto);
                    Imprimir mensaje de esperando a los demas jugadores;
                    Avisar al padre que el hijo esta listo;
                }
                Sino{
                    Aumentar la varible turno para llevar la cuenta de clientes conectados;
                    Cerrar conexion no utilizada;
                }
            }
        }
        Si(Estoy en el padre){
            Inicializar la mesa sin cartas;
            Esperar a los jugadores;
            Mientras(Queden cartas por repartir){
                Repartir 3 cartas a cada jugador;
                Si(Es la primera mano){
                    Repartir 4 cartas en la mesa;
                }
                Contar cuantas cartas hay repartidas;
                Imprimir el numero de ronda en el servidor;
                Para(Las 3 rondas){
                    Para(Los 4 jugadores){
                        Enviar a todos los jugadores sus cartas;
                        Enviar a los jugadores que toca jugar;
                        Si al jugador que le tocaba jugar levanto se imprime que fue el ultimo en levantar;
                        Enviar jugada a los demas jugadores;
                    }
                }
            }
            Asignar las cartas sobrantes en mesa al ultimo jugador en levantar;
            Enviar a todos los jugadores que finalizo la partida;
        }
        Sino{
            Inicializar cartas levantadas sin cartas;
            Hacer{
                Si(Se recibe 'A' del padre){
                    Contar cartas en la mesa;
                    Si(Quedan cartas sobre la mesa){
                        Enviar cuales son las cartas;
                    }
                    Sino{
                        Enviar que no hay cartas sobre la mesa;
                    }
                    Si(El jugador tiene cartas en la mano){
                        Enviar cuales son las cartas;
                    }
                    Sino{
                        Enviar que no tiene cartas en la mano;
                    }
                    Si(es el turno del jugador){
                        Imprimir que se espera la jugada;
                    }
                    Sino{
                        Imprimir de quien es el turno;
                    }
                }
                Si(Se recibe 'T' del padre){
                    Si(Es el turno del jugador){
                        Contar cartas en la mesa;
                        Si(Quedan cartas sobre la mesa){
                            Enviar cuales son las cartas;
                        }
                        Sino{
                            Enviar que no hay cartas sobre la mesa;
                        }
                        Si(El jugador tiene cartas en la mano){
                            Enviar cuales son las cartas;
                        }
                        Sino{
                            Enviar que no tiene cartas en la mano;
                        }
                        Mientras(La jugada no sea valida){
                            Si(Quedan cartas sobre la mesa){
                                Preguntar si el jugador quiere levantar o descartar;
                            }
                            Sino{
                                Decirle que descarte;
                            }
                            Si(Levanta){
                                Pedir que seleccione una carta de la mano y las correspondientes de la mesa;
                                Si(la suma no da 15){
                                    La jugada no es valida;
                                }
                                Sino{
                                    Enviar la jugada al padre;
                                    Contar cartas en la mesa;
                                    Si(No quedan cartas sobre la mesa){
                                        El jugador hizo escoba;
                                    }
                                    La jugada es valida;
                                }
                            }
                            Si(Descarta){
                                Pedir que seleccione una carta de la mano para descartar;
                                Enviar la jugada al padre;
                                La jugada es valida;
                            }
                        }
                    }
                }
                Si(Se recibe 'L' del padre){
                    Enviar la jugada de levante que hizo el jugador anterior;
                    Si (Hizo escoba){
                        Enviar que hizo escoba;
                    }
                }
                Si(Se recibe 'D' del padre){
                    Enviar la jugada de descarte que hizo el jugador anterior;
                }
            }Mientras(La partida no finalizo)
            Para(Las 4 jugadores){
                Enviar las cartas que levanto cada jugador y las escobas;
            }
            Liberar memoria compartida;
            return 0;
        }
        Cierre y eliminacion de memoria compartida y colas de mensajes
    FIN

\end{lstlisting}

\subsection{Compilación y ejecución del programa}

Con el código completo, para compilar el programa a un archivo binario ejecutable se llama al comando \texttt{gcc} (\textit{GNU Compiler Collection}):

\begin{center}
    \texttt{\$ gcc servidorEscoba.c -o servidorEscoba}\\
\end{center}

Con lo que se obtiene el archivo binario ejecutable \texttt{servidorEscoba}, el cual se ejecuta desde la terminal de la siguiente manera:

\begin{center}
    \texttt{\$ ./servidorEscoba}\\
\end{center}

Luego, se deben ejecutar los clientes utilizando el puerto 1234, ya sea con el comando \texttt{netcat} o \texttt{telnet}:

\begin{center}
    \texttt{\$ netcat \enquote{Direccion IP del servidor} 1234}\\
    \texttt{\$ telnet \enquote{Direccion IP del servidor} 1234}\\
\end{center}

Es importante ejecutar primero el servidor y luego los clientes.\\