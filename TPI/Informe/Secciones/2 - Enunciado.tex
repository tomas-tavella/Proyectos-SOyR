\section{Enunciado}
Queremos un programa \textbf{\textit{servidor concurrente}} que permita a un cliente que se conecta utilizando el protocolo TCP enviar un archivo para que el servidor lo almacene.\\

\begin{itemize}
    \item Cuando un cliente se conecta, el servidor enviará el mensaje: ``listo'' y esperará recibir la palabra ``archivo''.
    \item A continuación el servidor espera un espacio y luego el nombre con que se quiere guardar el archivo (solo letras, números y el carácter ``.'') finalizado con un espacio.
    \item Luego se quedará esperando un número codificado en \texttt{ascii} que indica el tamaño en bytes del archivo, también finalizado con un espacio.
    \item A continuación comenzará la recepción de los datos, que serán almacenados en un archivo cuyo nombre es el recibido en primer término, hasta completar la cantidad de bytes correspondiente.
    \item El servidor debe permitir la recepción de archivos binarios.
    \item Una vez recibida la totalidad de los datos, el servidor contestará con el siguiente mensaje:
    \begin{itemize}
        \item ``Archivo xx completo, tamaño declarado yy bytes, tamaño real zz bytes.''
        \begin{itemize}
            \item xx: nombre del archivo.
            \item yy: tamaño enviado en el mensaje del cliente.
            \item zz: total de bytes recibidos por el servidor.
        \end{itemize}
    \end{itemize}
    \item Luego el servidor cerrará la conexión con el cliente.
    \item El servidor debe llevar un archivo de registro de todas las conexiones entrantes, con fecha y hora de inicio y de finalización, tamaño del archivo recibido, cantidad de bytes enviados y cantidad de bytes recibidos en formato \texttt{csv}.
    \item Defina explícitamente cómo será el manejo de errores.
    \item Para probar el programa puede utilizar el cliente TCP que se vio como ejemplo, o el programa \texttt{nc}.
    \item Recuerde verificar que no queden procesos \textit{zombie} cuando finalizan los hijos generados.
\end{itemize}
